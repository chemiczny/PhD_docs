\begin{section}{Model QM I}
 W skład pierwszego modelu klasterowego KSTD weszło 196 atomów. Złożyły się na to następujące reszty:
 \begin{itemize}
  \item androst-4-en-3,17-dion
  \item FAD (fragment)
  \item GLY50-ALA51-SER52
  \item TYR119
  \item TYR318
  \item TYR487
  \item GLY491
 \end{itemize}

 Zamrożone atomy są widoczne na poniższych obrazkach jako powiększone sfery.
 
 \begin{figure}[H]
  \includegraphics[width=0.8\linewidth]{QM_I/s1.png}
\end{figure}

 \begin{figure}[H]
  \includegraphics[width=0.8\linewidth]{QM_I/s2.png}
\end{figure}
 
 Pierwszym modelowanym mechanizmem był ten zaproponowany w literaturze. Przeprowadzone serie obliczeń PES stosując ub3lyp/6-31G(d,p), tj jednowymiarowy skan odrywający
 wodór z atomu C2, a następnie kolejny jednowymiarowy skan odrywający wodór z atomu C1. Po analizie uzyskanych wyników okazało się, że 
 równie dobrze można stosować b3lyp/6-31G(d,p) -- macierze współczynników orbitali molekularnych $\alpha$ i $\beta$ były identyczne. Następnie dla znalezionych maksimów energii
 dla każdego ze skanów przeprowadzono obliczenia drgań, a uzyskane wyniki pozwoliły na znalezienie stanów przejściowych stosujac algorytm Berny'ego. Uzyskane struktury zostały
 zweryfikowane poprzez obliczenia drgań oraz IRC. Ostatecznie geometrie z obliczeń IRC zostały dodatkowo zoptymalizowane i poddane obliczeniom drgań -- by mieć pewność, że odpowiadają
 minimum lokalnym.
 
 Poniżej tabelaryczne zestawienie profilu reakcji:
 
  \begin{tabular}{||c c c c c c||} 
 \hline
 Metoda/Baza & TS1 & I z TS1 & I z TS2 & TS2 & P \\ [0.5ex] 
 \hline\hline
b3lyp/6-31G(d,p) & 13.96 &  0.16 & 0.15 & 14.54 & -6.09 \\
ZPE b3lyp/6-31G(d,p) & -2.37  & -0.27 &  -0.27  & -3.10 & 0.23 \\
PCM(eps = 4) b3lyp/6-31G(d,p) & 16.49  & 4.12 & 4.13 & 19.87 & 1.16 \\
b3lyp/6-311+G(2d,2p) & 14.17  & 0.06  & 0.06 & 14.90 & -7.90 \\
PCM(eps = 4) b3lyp/6-311+G(2d,2p) & 16.51  & 3.71 & 3.71  & 19.77 & -0.78 \\
PCM(eps = 4) b3lyp/6-311+G(2d,2p) & & & & & \\ 
+ ZPE b3lyp/6-31G(d,p)  & 14.14  & 3.44  & 3.44 & 16.66 & -0.54 \\
 \hline
 
\end{tabular}
 
 Do celów porównawczych przeprowadzono również obliczenia SP dla wybranych struktur z użyciem innych funkcjonałów (obliczenia w g09 więc wynik dla B3LYP również nieco się różni):
 
   \begin{tabular}{||c c c c  c||} 
 \hline
 Metoda/Baza & TS1 & I z TS1 & TS2 & P \\ [0.5ex] 
 \hline\hline
blyp/6-31G(d,p) & 13.88 & -2.01 & 11.54 & -3.14  \\
b3lyp/6-31G(d,p) & 14.23 &  0.30 & 14.85 & -5.91  \\
pbe/6-31G(d,p) & 10.90 & -2.33  & 7.43 & -2.83  \\
bp86/6-31G(d,p) & 10.12 & -3.75  & 7.21 & -3.08  \\
m05/6-31G(d,p) & 14.96 &  3.65 & 15.91 & -5.91  \\
m052x/6-31G(d,p) & 10.53 &  0.41 & 12.10 & -9.89  \\
m06l/6-31G(d,p) & 12.73 & 0.61  & 10.96 & 1.56  \\
m06/6-31G(d,p) & 10.16 &  -0.36 & 9.29 & -5.14  \\
m062x/6-31G(d,p) & 8.53 &  -1.18 & 11.33 & -8.62  \\
tpss/6-31G(d,p) & 13.50 & -0.79  & 12.32 & -1.23  \\
 \hline
 
\end{tabular}
 
 Oczywiście wyniki te byłyby bardziej wiarygodne gdyby przeprowadzić SP dla całego skanu, lub zoptymalizować geometrie dla każdego funkcjonału osobno (uwzględniając ZPE oraz
 rozpuszczalnik). Obliczenia zostały wykonane jedynie
 w celach poglądowych. Co ciekawe aż połowa z badanych funkcjonałów sugerowałaby, że etapem limitującym reakcje jest ten pierwszy. 
 
 Pliki z obliczeń:
 \path{/net/archive/groups/plggkatksdh/QM_models/QM_I/relaxed}
 
 \begin{subsection}{Model z cząsteczką wody}
  Biorąc pod uwagę nietypową sytuacje związaną ze zdeprotonową tyrozyną, rozważono możliwość, że istotnym czynnikiem stabilizującym
  ten układ może być cząsteczka (lub cząsteczki) wody. Aby zweryfikować tą hipotezę do przygotowanego modelu wprowadzono jedną
  cząsteczke wody. Celem sprawdzenia gdzie mogłaby ona być ulokowana sprawdzono dostępną strukturę krystaliczną pod kątem
  występowania wody w otoczeniu centrum aktywnego.
  
   \begin{figure}[H]
  \includegraphics[width=0.8\linewidth]{QM_I/4c3y_wat.png}
\end{figure}

 \begin{figure}[H]
  \includegraphics[width=0.8\linewidth]{QM_I/4c3y_wat2.png}
\end{figure}
  
  W obrębie wszystkich łańcuchów ze struktury krystalicznej, położenie cząsteczek wody względem centrum aktywnego wygląda niemal
  identycznie. Dlatego też umieszczono jedną taką cząsteczke pomiędzy pierścieniami TYR 318 a TYR 119 modelu QM pochodzącego 
  ze struktury po MD.
  
     \begin{figure}[H]
  \includegraphics[width=0.8\linewidth]{QM_I/wat1.png}
\end{figure}

 \begin{figure}[H]
  \includegraphics[width=0.8\linewidth]{QM_I/wat2.png}
\end{figure}
  
  Poniżej profil energetyczny reakcji uzyskany za pomocą powyższego modelu:
  
    \begin{tabular}{||c c c c c c||} 
 \hline
 Metoda/Baza & TS1 & I z TS1 & I z TS2 & TS2 & P \\ [0.5ex] 
 \hline\hline
b3lyp/6-31G(d,p) & 15.75 &  2.63 & 2.63 & 17.74 & -2.15 \\
ZPE b3lyp/6-31G(d,p) & -2.52  & -0.43 &  -0.43  & -3.38 & -0.23 \\
PCM(eps = 4) b3lyp/6-31G(d,p) & 18.11  & 6.26 & 6.26 & 22.83 & 4.60 \\
b3lyp/6-311+G(2d,2p) & 15.56  & 2.52  & 2.52 & 17.84 & -4.39 \\
PCM(eps = 4) b3lyp/6-311+G(2d,2p) & 17.62  & 5.75 & 5.75  & 22.37 & 2.16 \\
PCM(eps = 4) b3lyp/6-311+G(2d,2p) & & & & & \\ 
+ ZPE b3lyp/6-31G(d,p)  & 15.10  & 5.32  & 5.32 & 18.99 & 1.92 \\
 \hline
 
\end{tabular}

Porównując struktury tego modelu z jego odpowiednikiem bez cząsteczki wody można zauważyć bardzo drobne różnice w położeniach
atomów. Jak widać drugi etap reakcji uległ znacznie większej destabilizacji. 

Pliki z obliczeń:
\path{/net/archive/groups/plggkatksdh/QM_models/QM_I/relaxed_water}
  
 \end{subsection}

  
 \begin{subsection}{Model ze struktury krystalicznej}
 
 Przeprowadzono również analogiczne obliczenia wychodząc ze struktury krystalicznej 4c3y.
  
      \begin{tabular}{||c c c c c c||} 
 \hline
 Metoda/Baza & TS1 & I z TS1 & I z TS2 & TS2 & P \\ [0.5ex] 
 \hline\hline
b3lyp/6-31G(d,p) & 14.88 &  5.68 & 4.89 & 17.78 & 5.87 \\
ZPE b3lyp/6-31G(d,p) & -1.77  & 0.69 &  0.85  & -1.46 & 1.48 \\
PCM(eps = 4) b3lyp/6-31G(d,p) & 17.17  & 9.30 & 9.16 & 22.41 & 8.59 \\
b3lyp/6-311+G(2d,2p) & 15.75  & 7.19  & 7.55 & 20.63 & 5.93 \\
PCM(eps = 4) b3lyp/6-311+G(2d,2p) & 18.13  & 10.70 & 11.69  & 24.99 & 8.38 \\
PCM(eps = 4) b3lyp/6-311+G(2d,2p) & & & & & \\ 
+ ZPE b3lyp/6-31G(d,p)  & 16.36  & 11.39  & 12.53 & 23.53 & 9.85 \\
 \hline
 
\end{tabular}

W porównaniu do poprzednio omawianych modeli występują znaczne różnice w strukturach. Przede wszystkim:
\begin{itemize}
 \item odległość C$\alpha$ TYR 119 do C$\alpha$ TYR 487 w modelu z sieci krystalicznej wynosi około 7.4 \angstrom, natomiast w zrelaksowanym 9.9 \angstrom
 \item w przypadku modelu zrelaksowanego struktura produktu ulega znacznej deformacji: tworzy się wiązanie wodorowe pomiędzy TYR 487 a cząsteczką FAD, model z sieci krystalicznej zachowuje
 postać podobną do geometrii początkowej
\end{itemize}

  
  Pliki z obliczeń:
  \path{/net/archive/groups/plggkatksdh/QM_models/QM_I/crystal}
 \end{subsection}

 \begin{subsection}{Uwagi końcowe}
 
 Pierwszy model klasterowy miał szereg wad:
 \begin{itemize}
  \item bardzo duże różnice pomiędzy strukturą produktu, a pozostałymi
  \item wpływ rozpuszczalnika (modelowanego) jako PCM jest znaczny, szczególnie dla energii stanów przejściowych
  \item na końcowy profil energetyczny złożyło się kilka czynników, które niejednoznacznie wskazują, który etap jest faktycznie limitujący
 \end{itemize}
 
 Biorąc pod uwagę powyższe uwagi postanowiono powiększyć model -- powinno to pozwolić na zmniejszenie wpływu PCM na końcowy profil energetyczny.

  \end{subsection}
 
\end{section}
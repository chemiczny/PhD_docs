\begin{section}{Model QMMM II: fDYNAMO}
%środa
Wychodząc z wychłodzonego modelu po dynamice molekularnej przygotowano kolejny model QMMM. W warstwie MM zastosowano to samo pole AMBER, które było już wykorzystane
w modelowaniu QMMM ONIOM oraz MD. Do obszaru QM wybrano:

\begin{itemize}
 \item AND: androst-4-en-3,17-dion
 \item TYR318
 \item FAD
\end{itemize}


 \begin{figure}[H]
  \includegraphics[width=0.8\linewidth]{QMMM_fdynamo/highLayerDynamoRC.png}
\end{figure}

Jako ruchome atomy wybrano wszystkie reszty w promieniu 20 \angstrom od cząsteczki AND.

Po optymalizacji geometrii przeprowadzono dwuwymiarowy skan powierzchni energii potencjalnej amber:AM1 oraz amber:RM1. Wyniki sugerują to co poprzednie obliczenia: jedyny
prawdopodobny mechanizm  jest dwuetapowy i musi zaczynać się od transferu atomu wodory z atomu węgla C2.

 \begin{figure}[H]
  \includegraphics[width=0.8\linewidth]{QMMM_fdynamo/AM1_PES.png}
\end{figure}

 \begin{figure}[H]
  \includegraphics[width=0.8\linewidth]{QMMM_fdynamo/RM1_PES.png}
\end{figure}

Następnie wykorzystano geometrie ze zrelaksowanego skanu do znalezienia struktur TS1 oraz TS2 na poziomie amber:AM1 oraz amber:RM1. Struktury stanów przejściowych zostały zweryfikowane
przez obliczenia IRC -- wówczas okazało się, że metoda RM1 nie oddaje poprawnie struktury produktu przejściowego, dlatego też nie używano jej więcej do modelowania tej reakcji.

 \begin{figure}[H]
  \includegraphics[width=0.8\linewidth]{QMMM_fdynamo/IRM12.png}
\end{figure}


W kolejnym kroku przeprowadzono jednowymiarowe skany poprzez całą ścieżkę reakcji (amber:AM1), dla każdego z tych skanów obliczono również energie SP amber:b3lyp oraz amber:m062x.
 \begin{figure}[H]
  \includegraphics[width=0.8\linewidth]{QMMM_fdynamo/ts1_sp.png}
\end{figure}

 \begin{figure}[H]
  \includegraphics[width=0.8\linewidth]{QMMM_fdynamo/ts2_sp.png}
\end{figure}

Następnie przeprowadzono obliczenia PMF (amber:AM1) wykorzystując struktury z jednowymiarowych skanów. Parametry obliczeń PMF:
\begin{itemize}
 \item 303 K
 \item 40 ps 
 \item 40 000 kroków na okno
 \item 400 zapisanych sktruktur na okno
 \item 90 okien
\end{itemize}

Pełny profil energetyczny reakcji uzyskano poprzez WHAM (Weighted Histogram Analysis Method). Wykorzystując obliczenia SP dla amber:b3lyp oraz amber:m062x uzyskano poprawki spline.

W kolejnym etapie obliczeń znaleziono 10 struktur TS1 oraz 10 struktur TS2 (amber:AM1). Zostały one wybrane na podstawie różnicy między wartością ich współrzędnej reakcji oraz wartości współrzędnej reakcji
odpowiadającej maksimum energii w profilu PMF. Każda ze znalzionych geometrii stanów przejściowych została zweryfikowana przez obliczenia drgań oraz IRC (struktury po IRC również zostały zoptymalizowane
i poddane obliczeniom drgań).

Na tej podstawie przeprowadzono pierwsze obliczenia Kinetycznego Efektu Izotopowego.

\begin{subsection}{2D PES amber:DFT}
Korzystając ze struktur zoptymalizowanych podczas dwuwymiarowego skanu amber:AM1, przeprowadzono serie obliczeń SP z wykorzystaniem kilku funkcjonałów DFT.
Celem zmiejszenia czasu obliczeniowego, obszar QM został ograniczony do następującej struktury:

 \begin{figure}[H]
  \includegraphics[width=0.8\linewidth]{QMMM_fdynamo/smallQM_DFT_SP.png}
\end{figure}
 
\end{subsection}


\end{section}
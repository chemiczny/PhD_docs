\begin{section}{Model QMMM II: fDYNAMO}
%środa
Wychodząc z wychłodzonego modelu po dynamice molekularnej przygotowano kolejny model QMMM. W warstwie MM zastosowano to samo pole AMBER, które było już wykorzystane
w modelowaniu QMMM ONIOM oraz MD. Do obszaru QM wybrano:

\begin{itemize}
 \item AND: androst-4-en-3,17-dion
 \item TYR318
 \item FAD
\end{itemize}


 \begin{figure}[H]
  \includegraphics[width=0.8\linewidth]{QMMM_fdynamo/highLayerDynamoRC.png}
\end{figure}

Jako ruchome atomy wybrano wszystkie reszty w promieniu 20 \angstrom od cząsteczki AND.

Po optymalizacji geometrii przeprowadzono dwuwymiarowy skan powierzchni energii potencjalnej amber:AM1 oraz amber:RM1. Wyniki sugerują to co poprzednie obliczenia: jedyny
prawdopodobny mechanizm  jest dwuetapowy i musi zaczynać się od transferu atomu wodory z atomu węgla C2.

 \begin{figure}[H]
  \includegraphics[width=0.8\linewidth]{QMMM_fdynamo/AM1_AND_PES.png}
\end{figure}

 \begin{figure}[H]
  \includegraphics[width=0.8\linewidth]{QMMM_fdynamo/RM1_PES.png}
\end{figure}

Następnie wykorzystano geometrie ze zrelaksowanego skanu do znalezienia struktur TS1 oraz TS2 na poziomie amber:AM1 oraz amber:RM1. Struktury stanów przejściowych zostały zweryfikowane
przez obliczenia IRC -- wówczas okazało się, że metoda RM1 nie oddaje poprawnie struktury produktu przejściowego, dlatego też nie używano jej więcej do modelowania tej reakcji.

 \begin{figure}[H]
  \includegraphics[width=0.8\linewidth]{QMMM_fdynamo/IRM12.png}
\end{figure}


W kolejnym kroku przeprowadzono jednowymiarowe skany poprzez całą ścieżkę reakcji (amber:AM1), dla każdego z tych skanów obliczono również energie SP amber:b3lyp oraz amber:m062x.
 \begin{figure}[H]
  \includegraphics[width=0.8\linewidth]{QMMM_fdynamo/ts1_sp.png}
\end{figure}

 \begin{figure}[H]
  \includegraphics[width=0.8\linewidth]{QMMM_fdynamo/ts2_sp.png}
\end{figure}

Następnie przeprowadzono obliczenia PMF (amber:AM1) wykorzystując struktury z jednowymiarowych skanów. Parametry obliczeń PMF:
\begin{itemize}
 \item 303 K
 \item 40 ps 
 \item 40 000 kroków na okno
 \item 400 zapisanych sktruktur na okno
 \item 90 okien
\end{itemize}

Pełny profil energetyczny reakcji uzyskano poprzez WHAM (Weighted Histogram Analysis Method). Wykorzystując obliczenia SP dla amber:b3lyp oraz amber:m062x uzyskano poprawki spline.

W kolejnym etapie obliczeń znaleziono 10 struktur TS1 oraz 10 struktur TS2 (amber:AM1). Zostały one wybrane na podstawie różnicy między wartością ich współrzędnej reakcji oraz wartości współrzędnej reakcji
odpowiadającej maksimum energii w profilu PMF. Każda ze znalzionych geometrii stanów przejściowych została zweryfikowana przez obliczenia drgań oraz IRC (struktury po IRC również zostały zoptymalizowane
i poddane obliczeniom drgań).

Na tej podstawie przeprowadzono pierwsze obliczenia Kinetycznego Efektu Izotopowego, miały one jednak charakter wyłącznie orientacyjny, ze względu na inną strukturę substratów wykorzystanych do 
eksperymentów.

\begin{subsection}{2D PES amber:DFT}
Korzystając ze struktur zoptymalizowanych podczas dwuwymiarowego skanu amber:AM1, przeprowadzono serie obliczeń SP z wykorzystaniem funkcjonału B3LYP. Pod względem jakościowym mapa energii potencjalnej wygląda
bardzo podobnie jak w przypadku AM1 -- wygląda na to, że możliwe są tylko dwa dwuetapowe mechanizmy, z czego jeden z nich chcarakteryzuje się znacznie wyższą barierą niż drugi.

 \begin{figure}[H]
  \includegraphics[width=0.8\linewidth]{QMMM_fdynamo/B3LYP_AND_PES.png}
\end{figure}
 
\end{subsection}

\begin{subsection} { Pozostałe cząsteczki }

Przeprowadzono również analogiczne modelowanie dla cząsteczek, dla których przeprowadzone zostały eksperymenty związane z wyznaczeniem kinetycznego efektu izotopowego:

 \begin{figure}[H]
  \includegraphics[width=0.8\linewidth]{QMMM_fdynamo/subs2/structure.jpg}
   \caption{$2,2,4,6,6-d^{5}-17-$metylotestosteron​}
\end{figure}

 \begin{figure}[H]
  \includegraphics[width=0.8\linewidth]{QMMM_fdynamo/subs3/structure.jpg}
  \caption{ $1\alpha,16,16,17-d^{4}-$dihydrotestosteron }
\end{figure}
 
 Dla każdej z tych cząsteczek przeprowadzono obliczenia PES (analogicznie jak dla AND). Było to szczególnie istotne dla substratu bez podwójnego wiązania w pierścieniu A -- jak widać na poniższych obrazkach, 
 jakościowo oba wykresy wyglądają bardzo podobnie.
 
  \begin{figure}[H]
  \includegraphics[width=0.8\linewidth]{QMMM_fdynamo/subs2/pes.png}
   \caption{$2,2,4,6,6-d^{5}-17-$metylotestosteron​}
\end{figure}

 \begin{figure}[H]
  \includegraphics[width=0.8\linewidth]{QMMM_fdynamo/subs3/pes.png}
  \caption{ $1\alpha,16,16,17-d^{4}-$dihydrotestosteron }
\end{figure}
 
 Następnie dla każdego z tych dwóch substratów przeprowadzono obliczenia jednowymiarowych skanów energii wzdłuż każdej ze wsprółrzędnych reakcji. Wygenerowane struktury posłużyły do
 obliczeń PMF o parametrach:
 \begin{itemize}
 \item 303 K
 \item 40 ps 
 \item 40 000 kroków na okno
 \item 400 zapisanych sktruktur na okno
 \item 90 okien
\end{itemize}

Wyniki zostały przeanalizowane z wykorzystaniem metody WHAM, dzięki czemu uzyskano uśrednione profile energetyczne. Stąd wyznaczono współrzędne reakcji odpowiadające maksimum każdego z profilu.
Wybrano po 10 geometrii, które były najbliższe maksimum energii i dla nich przeprowadzono optymalizacje stanu przejściowego. Każdy ze znalezionych stanów przejściowych został zweryfikowany analizą
drgań oraz obliczeniami IRC. Na podstawie tych profili obliczono termodynamiczny kinetyczny efekt izotopowy.

Dla $2,2,4,6,6-d^{5}-17-$metylotestosteron​u wyniósł on:
 \begin{equation}
    KIE_{TS1} \left( \Delta  G^{QC}_{act} \right) = 4.801
  \end{equation}
  
  \begin{equation}
    KIE_{TS2} \left( \Delta  G^{QC}_{act} \right) = 1.107 
  \end{equation}
 
natomiast dla $1\alpha,16,16,17-d^{4}-$dihydrotestosteronu:
 \begin{equation}
    KIE_{TS1} \left( \Delta  G^{QC}_{act} \right) = 1.033
  \end{equation}
  
  \begin{equation}
    KIE_{TS2} \left( \Delta  G^{QC}_{act} \right) = 4.807 
  \end{equation}
  
Obliczono również BIE ( \textit{Binding Isotope Effect} ) dla obydwu tych związków. Do tego celu wykorzystano 10 zoptymalizowanych struktur QMMM zawierających cząsteczke substratu. Ograniczono obszar QM
do samego ligandu i zoptymalizowano całość jescze raz. Wybrano również 10 losowych struktur z symulacji dynamiki molekularnej ligandu w roztworze wodnym i również je zoptymalizowano metodą QMMM.

BIE dla $2,2,4,6,6-d^{5}-17-$metylotestosteron​u:
 \begin{equation}
    BIE \left( \Delta  G^{QC}_{act} \right) = 1.036
  \end{equation}
  
 BIE dla $1\alpha,16,16,17-d^{4}-$dihydrotestosteronu:
 \begin{equation}
    BIE \left( \Delta  G^{QC}_{act} \right) = 1.023
  \end{equation}
 
\end{subsection}


\end{section}
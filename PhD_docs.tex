\documentclass[10pt,a4paper]{article}
\usepackage[utf8]{inputenc}
\usepackage{amsmath}
\usepackage{amsfonts}
\usepackage{amssymb}
\usepackage{polski}
\usepackage[polish]{babel}
\usepackage{longtable}
\usepackage{url}
\usepackage{graphicx}
\usepackage{multicol}
\usepackage{float}
\usepackage{xcolor}
\usepackage{blindtext}
\title{Modelowanie mechanizmów reakcji dehydrogenaz 3-ketosteroidowych}
% \date{2017\\ December}
\author{Michał Glanowski\\ Instytut Katalizy i Fizykochemii Powierzchni PAN im. Jerzego Habera}
\graphicspath{ {./images/} }

\newcommand{\angstrom}{\mbox{\normalfont\AA}}

\begin{document}
\maketitle
\begin{section} {Porównanie struktur krystalicznych 4c3y oraz 4c3x}

Dehydrogenaza pochodząca z \textit{Rhodococus erythropolis} została skrytalizowana z produktem zadokowanym w centrum aktywnym oraz bez. Każda z tych struktur zawiera 2 tetramery 
białek. Poniżej krótkie zestawienie niektórych parametrów geometrycznych (wszystkie obliczone jako średnia z ośmiu struktur).

 \begin{tabular}{||c c c c c||} 
 \hline
 Struktura & Atom 1. & Atom 2. & Średnia odległość & Odchylenie standardowe \\ [0.5ex] 
 \hline\hline
4c3y & TYR 119 OH &  TYR 318 OH & 2.583 &0.008\\
4c3x & TYR 119 OH  & TYR 318 OH &  2.618 &0.004\\
4c3y & TYR 487 OH  & TYR 318 OH & 3.778 &0.005\\
4c3x & TYR 487 OH  & TYR 318 OH  & 3.261 &0.004\\
4c3y & TYR 487 OH  & TYR 119 OH & 5.689 &0.015\\
4c3x & TYR 487 OH  & TYR 119 OH  & 5.306 &0.004\\
4c3y & FAD N5 &  TYR 318 OH & 5.626 &0.063\\
4c3x & FAD N5 &  TYR 318 OH &  6.218 &0.048\\
4c3y & FAD N5 &  TYR 119 OH & 7.455 &0.052\\
4c3x & FAD N5 &  TYR 119 OH &  7.762 &0.063\\
4c3y & FAD N5 &  TYR 487 OH & 5.297 &0.061\\
4c3x & FAD N5 &  TYR 487 OH &   5.486 &0.026\\
4c3y & TYR 119 CA &  TYR 318 CA & 9.132 & 0.014\\
4c3x & TYR 119 CA &  TYR 318 CA  & 9.193 & 0.018\\
4c3y & TYR 487 CA &  TYR 318 CA & 8.893 & 0.010\\
4c3x & TYR 487 CA &  TYR 318 CA  & 8.962 & 0.016\\
4c3y & TYR 487 CA &  TYR 119 CA & 7.315 & 0.014\\
4c3x & TYR 487 CA &  TYR 119 CA  & 7.384 & 0.017\\
 \hline
 
\end{tabular}
 
\end{section}


\begin{section}{Dynamika molekularna KSTD I}

Cząsteczki występujące w modelu:
\begin{itemize}
 \item białko: 4c3y, pobrane z bazy PDB, łańcuch A
 \item FAD: geometria dostępna ze struktury krystalicznej
 \item AND: androst-4-en-3,17-dion, geometria na podstawie produktu zadokowanego w strukturze krystaliczenej
 \item TYR318: zdeprotonowana tyrozyna, ze względu na brak takiej reszty w oryginalnym polu siłowym amber konieczność parametryzacji
\end{itemize}

Wykorzystane oprogramowanie:
\begin{itemize}
 \item AMBER 16
 \item Gaussian16
 \item RESP
 \item propka
 \item H++
 \item serwer do przygotowywania białka do dynamiki: http://www.playmolecule.org/
 \item moduł pythonowy htmd
\end{itemize}


\begin{subsection}{Przygotowanie obliczeń}

\begin{subsubsection}{Białko}
 pKa aminokwasów zostało oszacowane za pomocą programów: propka3.1 oraz H++. Struktura wykorzystana w obliczeniach zawierała ligandy (FAD oraz AND). 
 
 Poniżej zestawienie wyników (posortowane rosnąco względem różnicy pKa):
 
%   \begin{tabular}{|| c c c c||} 
\begin{longtable}{| p{.25\textwidth} | p{.25\textwidth} | p{.25\textwidth} | p{.25\textwidth} |} 
 \hline
 Reszta & propka & H++ & różnica  \\ [0.5ex] 
 \hline\hline
GLU-87 & 4.22 & 4.170 & 0.04\\
GLU-375 & 2.57 & 2.631 & 0.06\\
ASP-83 & 3.87 & 3.942 & 0.07\\
GLU-215 & 4.65 & 4.727 & 0.07\\
GLU-71 & 3.27 & 3.178 & 0.09\\
GLU-201 & 4.67 & 4.575 & 0.09\\
ASP-138 & 3.43 & 3.319 & 0.11\\
ASP-305 & 3.36 & 3.248 & 0.11\\
GLU-393 & 4.68 & 4.562 & 0.11\\
ASP-330 & 4.01 & 4.131 & 0.12\\
GLU-123 & 4.68 & 4.546 & 0.13\\
GLU-240 & 4.59 & 4.454 & 0.13\\
GLU-7 & 4.04 & 4.182 & 0.14\\
GLU-405 & 4.6 & 4.454 & 0.14\\
ASP-397 & 3.95 & 3.802 & 0.14\\
ASP-470 & 3.04 & 2.862 & 0.17\\
GLU-364 & 4.38 & 4.193 & 0.18\\
ASP-202 & 3.91 & 3.704 & 0.20\\
ASP-372 & 3.93 & 3.702 & 0.22\\
ASP-161 & 2.77 & 2.535 & 0.23\\
GLU-433 & 4.81 & 4.561 & 0.24\\
ASP-144 & 3.25 & 3.510 & 0.25\\
ASP-289 & 2.81 & 2.542 & 0.26\\
GLU-110 & 4.75 & 5.039 & 0.28\\
GLU-62 & 4.73 & 4.425 & 0.30\\
ASP-141 & 4.34 & 4.666 & 0.32\\
HIS-162 & 5.52 & 5.894 & 0.37\\
LYS-510 & 10.63 & 10.246 & 0.38\\
LYS-361 & 10.72 & 11.133 & 0.41\\
GLU-406 & 4.02 & 3.599 & 0.42\\
LYS-186 & 10.38 & 10.803 & 0.42\\
GLU-345 & 4.05 & 3.613 & 0.43\\
GLU-192 & 4.87 & 4.405 & 0.46\\
ASP-386 & 3.52 & 3.992 & 0.47\\
LYS-400 & 10.55 & 11.122 & 0.57\\
ASP-154 & 4.31 & 3.723 & 0.58\\
GLU-411 & 4.8 & 5.387 & 0.58\\
ASP-277 & 3.62 & 3.029 & 0.59\\
GLU-94 & 4.13 & 3.500 & 0.62\\
ASP-465 & 2.19 & 2.823 & 0.63\\
GLU-376 & 4.79 & 4.155 & 0.63\\
ASP-9 & 2.89 & 2.162 & 0.72\\
TYR-502 & 10.93 & 11.684 & 0.75\\
LYS-148 & 10.64 & 11.396 & 0.75\\
GLU-237 & 4.28 & 5.044 & 0.76\\
GLU-188 & 4.05 & 3.245 & 0.80\\
GLU-209 & 4.79 & 3.967 & 0.82\\
ASP-127 & 4.0 & 3.142 & 0.85\\
LYS-121 & 10.57 & 11.446 & 0.87\\
ASP-136 & 3.85 & 2.971 & 0.87\\
ASP-326 & 4.94 & 4.012 & 0.92\\
GLU-309 & 5.2 & 6.151 & 0.95\\
LYS-380 & 10.45 & 11.412 & 0.96\\
ASP-329 & 3.39 & 2.427 & 0.96\\
GLU-112 & 4.21 & 3.234 & 0.97\\
GLU-211 & 4.2 & 5.185 & 0.98\\
ASP-40 & 3.81 & 2.807 & 1.00\\
LYS-247 & 10.14 & 11.241 & 1.10\\
ASP-507 & 3.93 & 2.678 & 1.25\\
ASP-261 & 3.07 & 1.728 & 1.34\\
GLU-286 & 4.64 & 3.127 & 1.51\\
HIS-408 & 6.84 & 8.407 & 1.56\\
GLU-104 & 2.84 & 1.269 & 1.57\\
ASP-90 & 2.39 & 0.691 & 1.69\\
GLU-85 & 4.06 & 2.323 & 1.73\\
ASP-415 & 5.68 & 3.872 & 1.80\\
HIS-362 & 4.3 & 6.239 & 1.93\\
HIS-328 & 6.97 & 9.197 & 2.22\\
ASP-68 & 2.55 & 0.301 & 2.24\\
ASP-342 & 3.11 & 0.837 & 2.27\\
ASP-456 & 3.64 & 0.561 & 3.07\\
GLU-152 & 4.99 & 1.029 & 3.96\\
ARG-344 & 12.47 & $>$12.000 & -\\
ARG-299 & 13.5 & $>$12.000 & -\\
TYR-414 & 14.02 & $>$12.000 & -\\
TYR-24 & 10.45 & $>$12.000 & -\\
TYR-473 & 13.47 & $>$12.000 & -\\
ASP-412 & 2.87 & $<$0.000 & -\\
ASP-446 & 4.11 & $<$0.000 & -\\
ARG-239 & 12.77 & $>$12.000 & -\\
ARG-150 & 13.86 & $>$12.000 & -\\
TYR-318 & 22.84 & $>$12.000 & -\\
ARG-176 & 11.28 & $>$12.000 & -\\
ARG-323 & 13.07 & $>$12.000 & -\\
CYS-282 & 13.47 & $>$12.000 & -\\
ARG-310 & 15.29 & $>$12.000 & -\\
ASP-118 & 5.33 & $<$0.000 & -\\
CYS-8 & 12.04 & $>$12.000 & -\\
GLU-233 & 6.22 & $<$0.000 & -\\
ARG-63 & 13.98 & $>$12.000 & -\\
TYR-119 & 17.57 & $>$12.000 & -\\
TYR-438 & 10.14 & $>$12.000 & -\\
ASP-404 & 1.74 & $<$0.000 & -\\
ARG-41 & 12.89 & $>$12.000 & -\\
ARG-114 & 13.08 & $>$12.000 & -\\
ARG-389 & 14.35 & $>$12.000 & -\\
TYR-120 & 12.7 & $>$12.000 & -\\
TYR-48 & 13.05 & $>$12.000 & -\\
ARG-223 & 12.6 & $>$12.000 & -\\
ARG-463 & 12.33 & $>$12.000 & -\\
ARG-204 & 13.05 & $>$12.000 & -\\
CYS-353 & 11.83 & $>$12.000 & -\\
TYR-487 & 14.21 & $>$12.000 & -\\
ARG-125 & 13.69 & $>$12.000 & -\\
GLU-37 & 3.15 & $<$0.000 & -\\
TYR-311 & 12.66 & $>$12.000 & -\\
ARG-218 & 12.34 & $>$12.000 & -\\
ARG-78 & 13.15 & $>$12.000 & -\\
ARG-74 & 15.6 & $>$12.000 & -\\
ARG-171 & 10.56 & $>$12.000 & -\\
ARG-409 & 11.51 & $>$12.000 & -\\
LYS-38 & 10.13 & $>$12.000 & -\\
LYS-220 & 10.54 & $>$12.000 & -\\
ASP-156 & 3.57 & $<$0.000 & -\\
LYS-450 & 6.49 & $>$12.000 & -\\
ARG-88 & 11.39 & $>$12.000 & -\\
ARG-503 & 11.77 & $>$12.000 & -\\
ARG-441 & 12.83 & $>$12.000 & -\\
TYR-76 & 13.96 & $>$12.000 & -\\
GLU-314 & 3.46 & $<$0.000 & -\\
CYS-419 & 10.57 & $>$12.000 & -\\
ARG-157 & 10.86 & $>$12.000 & -\\
LYS-394 & 11.38 & $>$12.000 & -\\
ARG-460 & 13.64 & $>$12.000 & -\\
ASP-319 & 4.46 & $<$0.000 & -\\
ARG-130 & 14.06 & $>$12.000 & -\\
ARG-190 & 12.74 & $>$12.000 & -\\
ARG-485 & 12.34 & $>$12.000 & -\\
TYR-92 & 14.51 & $>$12.000 & -\\
 \hline
 \end{longtable}
% \end{tabular}
 
 Należy zauważyć, że w przypadku TYR318 propka sugeruje pKa na poziomie 22.84 -- jest to sprzeczne z postulowanym mechanizme, ale należy pamiętać, że propka 
 uwzględnia ligandy w obliczeniach w sposób bardzo przybliżony. Przewidywane pI przez program propKa (4.40) jest w bardzo dobrej zgodności z eksperymentalnym (4.70).
 
 Aby ostatecznie przygotować cząsteczke białka do symulacji MD posłużono się narzędziem: PlayMolecule ProteinPrepare (publikacja: "PlayMolecule ProteinPrepare:
 A Web Application for Protein Preparation for Molecular Dynamics Simulations." Martinez-Rosell, Giorgino, De Fabritiis ).
 Narzędzie to wykorzystuje propke oraz PDB2PQR (optymalizacja wiązań wodorowych w białku) jako programy pomocnicze. Korzystając z tego i wyników uzyskanych niezależnie z propki i H++
 ustalono stany protonacyjne aminokwasów w cząsteczce.
 
 Istotne pliki:
 
\begin{itemize}
 \item \path{/home/glanowski/docs/KSTD_MD_I/proteinPrepare/monomer_substrate_final.pdb} -- struktura łańcucha A 4c3y
 \item \path{/home/glanowski/docs/KSTD_MD_I/proteinPrepare/monomer_substrate_final.pka} -- wyniki z propKi
 \item \path{/home/glanowski/docs/KSTD_MD_I/proteinPrepare/h++.log} -- wyniki z H++
 \item \path{/home/glanowski/docs/KSTD_MD_I/proteinPrepare/propKavsHpp.py} -- skrypt do porównania wyników
\end{itemize}

\end{subsubsection}

\begin{subsubsection}{FAD}

Ładunki dla cząsteczki FAD zostąły pobrane z bazy R.E.DD.B. (https://upjv.q4md-forcefieldtools.org/REDDB/projects/F-91/, kopia projektu: \path{/home/glanowski/docs/KSTD_MD_I/FAD/F-91} ).
Szczegóły obliczeniowe wg autorów:

\begin{verbatim}
Geometry optimization

Program 1 GAUSSIAN 2003
Theory level 1 HF
More information 1 Tight
Basis set 1 6-31G*

Molecular electrostatic potential computation

Program 2 GAUSSIAN 2003
Theory level 2 DFT B3LYP
More information 2 IOp(6/33=2) SCRF(IEFPCM,Solvent=Ether) NoSymm
Basis set 2 cc-pVTZ

Algorithm CONNOLLY SURFACE

Information about the charge fit

Program RESP
Number of stage(s) 2
\end{verbatim}

Po pierwszym uruchomieniu obliczeń wprowadzono jedną modyfikacje w pliku frcmod: zwiększono wartość improper:\\
\begin{verbatim}
 h5-nb-ca-nb        10.5          180.0         2.0
\end{verbatim}
aby zapobiec wygięciu pierścieni w cząsteczce FAD.

Ładunki cząstkowe atomów w cząsteczce FAD:
 \begin{figure}[H]
  \includegraphics[width=0.8\linewidth]{MD_KSTD_I/FAD/FAD_charges1.png}
\end{figure}

 \begin{figure}[H]
  \includegraphics[width=0.8\linewidth]{MD_KSTD_I/FAD/FAD_charges2.png}
\end{figure}

 \begin{figure}[H]
  \includegraphics[width=0.8\linewidth]{MD_KSTD_I/FAD/FAD_charges3.png}
\end{figure}

 \begin{figure}[H]
  \includegraphics[width=0.8\linewidth]{MD_KSTD_I/FAD/FAD_charges4.png}
\end{figure}

 \begin{figure}[H]
  \includegraphics[width=0.8\linewidth]{MD_KSTD_I/FAD/FAD_charges5.png}
\end{figure}
 
\end{subsubsection}

\begin{subsubsection}{AND}
Cząsteczka androst-4-en-3,17-dionu została zoptymalizowana w próżni (gaussian09 b3lyp/6-31g(d)). Dla zoptymalizowanej struktury
przeprowadzono obliczenia ESP. Następnie za pomocą programu antechamber(RESP) przekonwertowano uzyskany plik gesp na plik mol2, zwierający ładunki atomów.
Utworzono również pliki prepi oraz frcmod.
Pliki z obliczeń:
\path{/home/glanowski/docs/KSTD_MD_I/AND}

 
\end{subsubsection}

\begin{subsubsection}{Zdeprotonowana tyrozyna}

Aby obliczyć ładunki zdeprotonowanej tyrozyny, zoptymalizowano geometrie (gaussian09 b3lyp/6-31g(d)) następującej cząsteczki:

\begin{figure}[H]
  \includegraphics[width=0.5\linewidth]{MD_KSTD_I/TYM/tym_capped.png}
\end{figure}

Następnie wykorzystano program RESP z opcją wymuszenia ładunków na N i C końcu peptydu (w ten sposób sumaryczny ładunek reszty będzie równy dokładnie -1):

\begin{figure}[H]
  \includegraphics[width=0.5\linewidth]{MD_KSTD_I/TYM/ace_with_charges.jpg}
\end{figure}

\begin{figure}[H]
  \includegraphics[width=0.5\linewidth]{MD_KSTD_I/TYM/nme_with_charges.jpg}
\end{figure}

Ładunki cząstkowe na reszcie zdeprotonowanej tyrozyny:
\begin{figure}[H]
  \includegraphics[width=0.8\linewidth]{MD_KSTD_I/TYM/charges.png}
\end{figure}

Warto zauważyć, że ładunki na równocennych atomach są identyczne -- czyli tak jak powinno być.

Źródło: \path{http://ambermd.org/tutorials/advanced/tutorial1/section1.htm}
Pliki z obliczeń:
\path{/home/glanowski/docs/KSTD_MD_I/TYM}

\end{subsubsection}

 
\end{subsection}

\begin{subsection}{Obliczenia}
Cały model KSTD został najpierw zminimalizowany w 5 krokach. W pierwszych 4 krokach na atomy białka i ligandów został nałożony dodatkowy potencjał mający ograniczyć ich ruch, 
tak aby najpierw zoptymalizować wygenerowane cząsteczki wody. Stałe siłowe wynosiły kolejno: 500, 250, 100 i 10 $\frac{kcal}{mol \cdot \angstrom^2 }$.

Następnie model został ogrzany do 303 K w czasie 100 ps, w warunkach stałej objętości. Na białko i ligandy nałożony został nieznaczny dodatkowy potencjał
1 $\frac{kcal}{mol \cdot \angstrom^2 }$ aby zabezpieczyć ich strukturę. W kolejnym kroku przeprowadzono 100 ps dynamiki pod stałym ciśnieniem z identycznym dodatkowym potencjałem, w 
temperaturze 303 K. W ten sposób przygotowano model do właściwej symulacji dynamiki molekularnej (stałe ciśnienie, 303 K).

Ostatecznie przeprowadzono 132 ns symulacji, jednakże już po upływie 45 ns wybrano jej fragment do dalszej analizy. 20 nanosekund (oznaczone zieloną ramką) poddano analizie skupień.
Zastosowano metodę k-średnich. Do inicjalizacji analizy skupień wybrano w sposób losowy co dziesiątą klatke. 
Liczba klastrów wynosiła 5, ostatecznie wybrano strukture leżącą najbliżej centroidu najczęśćiej występującego skupiska.



\begin{figure}[H]
  \includegraphics[width=0.5\linewidth]{MD_KSTD_I/rmsdGutSel.png}
\end{figure}

\end{subsection}

\end{section}

\begin{section}{Uwagi dotyczące przechowywania plików z gaussiana}

Wszystkie dane pochodzące z obliczeń zostały zorganizowane w analogiczny sposób: w katalogu dotyczacym danego mechanizmu/modelu znajdują się katalogi nazwane kolejno TS1, TS2...
W każdym katalogu, który dotyczy obliczeń pojedynczego stany przejściowego znajdują się następujące podkatalogi:
\begin{itemize}
 \item \path{freq_verify} -- zawiera obliczenia drgań dla struktury znalezionego stanu przejściowego, ponadto zawiera podfoldery zawierające obliczenia SP
 \item \path{irc_reverse} -- obliczenia IRC dla znalezionej struktury stanu przejściowego (tylko w jedną strone wektora odpowiadającemu urojonej częstotliwości), w jego wnętrzu znajduje
 się podkatalog opt/ (optymalizacja geometrii), a w jego wnętrzu podkatalogi odpowiadające obliczeniom SP oraz drgań dla zoptymalizowanej struktury
 \item \path{irc_forward} -- analogicznie jak dla irc reverse/ z tą różnicą, że tutaj IRC zostało wykonane w przeciwną stronę.
 \item \path{dumped_structures} -- zawiera struktury: stanu przejściowego, oraz zoptymalizowane struktury produktu i substratu (dla tego etapu reakcji) w postaci plików wejściowych Gaussiana
 \item pozostałe katalogi zawierają obliczenia skanów itd.
\end{itemize}

 
\end{section}


\begin{section}{Model QM I}
 W skład pierwszego modelu klasterowego KSTD weszło 196 atomów. Złożyły się na to następujące reszty:
 \begin{itemize}
  \item androst-4-en-3,17-dion
  \item FAD (fragment)
  \item GLY50-ALA51-SER52
  \item TYR119
  \item TYR318
  \item TYR487
  \item GLY491
 \end{itemize}

 Zamrożone atomy są widoczne na poniższych obrazkach jako powiększone sfery.
 
 \begin{figure}[H]
  \includegraphics[width=0.8\linewidth]{QM_I/s1.png}
\end{figure}

 \begin{figure}[H]
  \includegraphics[width=0.8\linewidth]{QM_I/s2.png}
\end{figure}
 
 Pierwszym modelowanym mechanizmem był ten zaproponowany w literaturze. Przeprowadzone serie obliczeń PES stosując ub3lyp/6-31G(d,p), tj jednowymiarowy skan odrywający
 wodór z atomu C2, a następnie kolejny jednowymiarowy skan odrywający wodór z atomu C1. Po analizie uzyskanych wyników okazało się, że 
 równie dobrze można stosować b3lyp/6-31G(d,p) -- macierze współczynników orbitali molekularnych $\alpha$ i $\beta$ były identyczne. Następnie dla znalezionych maksimów energii
 dla każdego ze skanów przeprowadzono obliczenia drgań, a uzyskane wyniki pozwoliły na znalezienie stanów przejściowych stosujac algorytm Berny'ego. Uzyskane struktury zostały
 zweryfikowane poprzez obliczenia drgań oraz IRC. Ostatecznie geometrie z obliczeń IRC zostały dodatkowo zoptymalizowane i poddane obliczeniom drgań -- by mieć pewność, że odpowiadają
 minimum lokalnym.
 
 Poniżej tabelaryczne zestawienie profilu reakcji:
 
  \begin{tabular}{||c c c c c c||} 
 \hline
 Metoda/Baza & TS1 & I z TS1 & I z TS2 & TS2 & P \\ [0.5ex] 
 \hline\hline
b3lyp/6-31G(d,p) & 13.96 &  0.16 & 0.15 & 14.54 & -6.09 \\
ZPE b3lyp/6-31G(d,p) & -2.37  & -0.27 &  -0.27  & -3.10 & 0.23 \\
PCM(eps = 4) b3lyp/6-31G(d,p) & 16.49  & 4.12 & 4.13 & 19.87 & 1.16 \\
b3lyp/6-311+G(2d,2p) & 14.17  & 0.06  & 0.06 & 14.90 & -7.90 \\
PCM(eps = 4) b3lyp/6-311+G(2d,2p) & 16.51  & 3.71 & 3.71  & 19.77 & -0.78 \\
PCM(eps = 4) b3lyp/6-311+G(2d,2p) & & & & & \\ 
+ ZPE b3lyp/6-31G(d,p)  & 14.14  & 3.44  & 3.44 & 16.66 & -0.54 \\
 \hline
 
\end{tabular}
 
 Do celów porównawczych przeprowadzono również obliczenia SP dla wybranych struktur z użyciem innych funkcjonałów (obliczenia w g09 więc wynik dla B3LYP również nieco się różni):
 
   \begin{tabular}{||c c c c  c||} 
 \hline
 Metoda/Baza & TS1 & I z TS1 & TS2 & P \\ [0.5ex] 
 \hline\hline
blyp/6-31G(d,p) & 13.88 & -2.01 & 11.54 & -3.14  \\
b3lyp/6-31G(d,p) & 14.23 &  0.30 & 14.85 & -5.91  \\
pbe/6-31G(d,p) & 10.90 & -2.33  & 7.43 & -2.83  \\
bp86/6-31G(d,p) & 10.12 & -3.75  & 7.21 & -3.08  \\
m05/6-31G(d,p) & 14.96 &  3.65 & 15.91 & -5.91  \\
m052x/6-31G(d,p) & 10.53 &  0.41 & 12.10 & -9.89  \\
m06l/6-31G(d,p) & 12.73 & 0.61  & 10.96 & 1.56  \\
m06/6-31G(d,p) & 10.16 &  -0.36 & 9.29 & -5.14  \\
m062x/6-31G(d,p) & 8.53 &  -1.18 & 11.33 & -8.62  \\
tpss/6-31G(d,p) & 13.50 & -0.79  & 12.32 & -1.23  \\
 \hline
 
\end{tabular}
 
 Oczywiście wyniki te byłyby bardziej wiarygodne gdyby przeprowadzić SP dla całego skanu, lub zoptymalizować geometrie dla każdego funkcjonału osobno (uwzględniając ZPE oraz
 rozpuszczalnik). Obliczenia zostały wykonane jedynie
 w celach poglądowych. Co ciekawe aż połowa z badanych funkcjonałów sugerowałaby, że etapem limitującym reakcje jest ten pierwszy. 
 
 Pliki z obliczeń:
 \path{/net/archive/groups/plggkatksdh/QM_models/QM_I/relaxed}
 
 \begin{subsection}{Model z cząsteczką wody}
  Biorąc pod uwagę nietypową sytuacje związaną ze zdeprotonową tyrozyną, rozważono możliwość, że istotnym czynnikiem stabilizującym
  ten układ może być cząsteczka (lub cząsteczki) wody. Aby zweryfikować tą hipotezę do przygotowanego modelu wprowadzono jedną
  cząsteczke wody. Celem sprawdzenia gdzie mogłaby ona być ulokowana sprawdzono dostępną strukturę krystaliczną pod kątem
  występowania wody w otoczeniu centrum aktywnego.
  
   \begin{figure}[H]
  \includegraphics[width=0.8\linewidth]{QM_I/4c3y_wat.png}
\end{figure}

 \begin{figure}[H]
  \includegraphics[width=0.8\linewidth]{QM_I/4c3y_wat2.png}
\end{figure}
  
  W obrębie wszystkich łańcuchów ze struktury krystalicznej, położenie cząsteczek wody względem centrum aktywnego wygląda niemal
  identycznie. Dlatego też umieszczono jedną taką cząsteczke pomiędzy pierścieniami TYR 318 a TYR 119 modelu QM pochodzącego 
  ze struktury po MD.
  
     \begin{figure}[H]
  \includegraphics[width=0.8\linewidth]{QM_I/wat1.png}
\end{figure}

 \begin{figure}[H]
  \includegraphics[width=0.8\linewidth]{QM_I/wat2.png}
\end{figure}
  
  Poniżej profil energetyczny reakcji uzyskany za pomocą powyższego modelu:
  
    \begin{tabular}{||c c c c c c||} 
 \hline
 Metoda/Baza & TS1 & I z TS1 & I z TS2 & TS2 & P \\ [0.5ex] 
 \hline\hline
b3lyp/6-31G(d,p) & 15.75 &  2.63 & 2.63 & 17.74 & -2.15 \\
ZPE b3lyp/6-31G(d,p) & -2.52  & -0.43 &  -0.43  & -3.38 & -0.23 \\
PCM(eps = 4) b3lyp/6-31G(d,p) & 18.11  & 6.26 & 6.26 & 22.83 & 4.60 \\
b3lyp/6-311+G(2d,2p) & 15.56  & 2.52  & 2.52 & 17.84 & -4.39 \\
PCM(eps = 4) b3lyp/6-311+G(2d,2p) & 17.62  & 5.75 & 5.75  & 22.37 & 2.16 \\
PCM(eps = 4) b3lyp/6-311+G(2d,2p) & & & & & \\ 
+ ZPE b3lyp/6-31G(d,p)  & 15.10  & 5.32  & 5.32 & 18.99 & 1.92 \\
 \hline
 
\end{tabular}

Porównując struktury tego modelu z jego odpowiednikiem bez cząsteczki wody można zauważyć bardzo drobne różnice w położeniach
atomów. Jak widać drugi etap reakcji uległ znacznie większej destabilizacji. 

Pliki z obliczeń:
\path{/net/archive/groups/plggkatksdh/QM_models/QM_I/relaxed_water}
  
 \end{subsection}

  
 \begin{subsection}{Model ze struktury krystalicznej}
 
 Przeprowadzono również analogiczne obliczenia wychodząc ze struktury krystalicznej 4c3y.
  
      \begin{tabular}{||c c c c c c||} 
 \hline
 Metoda/Baza & TS1 & I z TS1 & I z TS2 & TS2 & P \\ [0.5ex] 
 \hline\hline
b3lyp/6-31G(d,p) & 14.88 &  5.68 & 4.89 & 17.78 & 5.87 \\
ZPE b3lyp/6-31G(d,p) & -1.77  & 0.69 &  0.85  & -1.46 & 1.48 \\
PCM(eps = 4) b3lyp/6-31G(d,p) & 17.17  & 9.30 & 9.16 & 22.41 & 8.59 \\
b3lyp/6-311+G(2d,2p) & 15.75  & 7.19  & 7.55 & 20.63 & 5.93 \\
PCM(eps = 4) b3lyp/6-311+G(2d,2p) & 18.13  & 10.70 & 11.69  & 24.99 & 8.38 \\
PCM(eps = 4) b3lyp/6-311+G(2d,2p) & & & & & \\ 
+ ZPE b3lyp/6-31G(d,p)  & 16.36  & 11.39  & 12.53 & 23.53 & 9.85 \\
 \hline
 
\end{tabular}

W porównaniu do poprzednio omawianych modeli występują znaczne różnice w strukturach. Przede wszystkim:
\begin{itemize}
 \item odległość C$\alpha$ TYR 119 do C$\alpha$ TYR 487 w modelu z sieci krystalicznej wynosi około 7.4 \angstrom, natomiast w zrelaksowanym 9.9 \angstrom
 \item w przypadku modelu zrelaksowanego struktura produktu ulega znacznej deformacji: tworzy się wiązanie wodorowe pomiędzy TYR 487 a cząsteczką FAD, model z sieci krystalicznej zachowuje
 postać podobną do geometrii początkowej
\end{itemize}

  
  Pliki z obliczeń:
  \path{/net/archive/groups/plggkatksdh/QM_models/QM_I/crystal}
 \end{subsection}

 \begin{subsection}{Uwagi końcowe}
 
 Pierwszy model klasterowy miał szereg wad:
 \begin{itemize}
  \item bardzo duże różnice pomiędzy strukturą produktu, a pozostałymi
  \item wpływ rozpuszczalnika (modelowanego) jako PCM jest znaczny, szczególnie dla energii stanów przejściowych
  \item na końcowy profil energetyczny złożyło się kilka czynników, które niejednoznacznie wskazują, który etap jest faktycznie limitujący
 \end{itemize}
 
 Biorąc pod uwagę powyższe uwagi postanowiono powiększyć model -- powinno to pozwolić na zmniejszenie wpływu PCM na końcowy profil energetyczny.

  \end{subsection}
 
\end{section}

\begin{section}{Model QM II}
%piątek
Bazując na dotychczasowych doświadczeniach powiększono model klasterowy. Reszty, które weszły w skład powiększonego modelu:
 \begin{itemize}
  \item androst-4-en-3,17-dion
  \item FAD (fragment)
  \item GLY50-ALA51-SER52
  \item {\color{green} ARG114}
  \item {\color{green} PHE116}
  \item TYR119
  \item {\color{green} PHE294}
  \item TYR318
  \item TYR487
  \item GLY491
 \end{itemize}

 \begin{figure}[H]
  \includegraphics[width=0.8\linewidth]{QM_II/model.png}
\end{figure}

 
\end{section}


\begin{section}{Model QMMM I: ONIOM}
%wtorek
 
\end{section}

\begin{section}{Model QMMM II: fDYNAMO}
%środa
Wychodząc z wychłodzonego modelu po dynamice molekularnej przygotowano kolejny model QMMM. W warstwie MM zastosowano to samo pole AMBER, które było już wykorzystane
w modelowaniu QMMM ONIOM oraz MD. Do obszaru QM wybrano:

\begin{itemize}
 \item AND: androst-4-en-3,17-dion
 \item TYR318
 \item FAD
\end{itemize}


 \begin{figure}[H]
  \includegraphics[width=0.8\linewidth]{QMMM_fdynamo/highLayerDynamoRC.png}
\end{figure}

Jako ruchome atomy wybrano wszystkie reszty w promieniu 20 \angstrom od cząsteczki AND.

Po optymalizacji geometrii przeprowadzono dwuwymiarowy skan powierzchni energii potencjalnej amber:AM1 oraz amber:RM1. Wyniki sugerują to co poprzednie obliczenia: jedyny
prawdopodobny mechanizm  jest dwuetapowy i musi zaczynać się od transferu atomu wodory z atomu węgla C2.

 \begin{figure}[H]
  \includegraphics[width=0.8\linewidth]{QMMM_fdynamo/AM1_PES.png}
\end{figure}

 \begin{figure}[H]
  \includegraphics[width=0.8\linewidth]{QMMM_fdynamo/RM1_PES.png}
\end{figure}

Następnie wykorzystano geometrie ze zrelaksowanego skanu do znalezienia struktur TS1 oraz TS2 na poziomie amber:AM1 oraz amber:RM1. Struktury stanów przejściowych zostały zweryfikowane
przez obliczenia IRC -- wówczas okazało się, że metoda RM1 nie oddaje poprawnie struktury produktu przejściowego, dlatego też nie używano jej więcej do modelowania tej reakcji.

 \begin{figure}[H]
  \includegraphics[width=0.8\linewidth]{QMMM_fdynamo/IRM12.png}
\end{figure}


W kolejnym kroku przeprowadzono jednowymiarowe skany poprzez całą ścieżkę reakcji (amber:AM1), dla każdego z tych skanów obliczono również energie SP amber:b3lyp oraz amber:m062x.
 \begin{figure}[H]
  \includegraphics[width=0.8\linewidth]{QMMM_fdynamo/ts1_sp.png}
\end{figure}

 \begin{figure}[H]
  \includegraphics[width=0.8\linewidth]{QMMM_fdynamo/ts2_sp.png}
\end{figure}

Następnie przeprowadzono obliczenia PMF (amber:AM1) wykorzystując struktury z jednowymiarowych skanów. Parametry obliczeń PMF:
\begin{itemize}
 \item 303 K
 \item 40 ps 
 \item 40 000 kroków na okno
 \item 400 zapisanych sktruktur na okno
 \item 90 okien
\end{itemize}

Pełny profil energetyczny reakcji uzyskano poprzez WHAM (Weighted Histogram Analysis Method). Wykorzystując obliczenia SP dla amber:b3lyp oraz amber:m062x uzyskano poprawki spline.

W kolejnym etapie obliczeń znaleziono 10 struktur TS1 oraz 10 struktur TS2 (amber:AM1). Zostały one wybrane na podstawie różnicy między wartością ich współrzędnej reakcji oraz wartości współrzędnej reakcji
odpowiadającej maksimum energii w profilu PMF. Każda ze znalzionych geometrii stanów przejściowych została zweryfikowana przez obliczenia drgań oraz IRC (struktury po IRC również zostały zoptymalizowane
i poddane obliczeniom drgań).

Na tej podstawie przeprowadzono pierwsze obliczenia Kinetycznego Efektu Izotopowego.

\begin{subsection}{2D PES amber:DFT}
Korzystając ze struktur zoptymalizowanych podczas dwuwymiarowego skanu amber:AM1, przeprowadzono serie obliczeń SP z wykorzystaniem kilku funkcjonałów DFT.
Celem zmiejszenia czasu obliczeniowego, obszar QM został ograniczony do następującej struktury:

 \begin{figure}[H]
  \includegraphics[width=0.8\linewidth]{QMMM_fdynamo/smallQM_DFT_SP.png}
\end{figure}
 
\end{subsection}


\end{section}

\begin{section}{Model QM III}
 Najistotniejszą zmianą w stosunku do poprzedniego modelu klasterowego KSTD było uwzględnienie fragmentu (wiązanie peptydowe) LEU494, który mógłby stabilizować cząsteczke FADu (zwłaszcza po przyłączeniu jonu
 $H^{-}$). Wzięto pod uwage również nieco więcej atomów z reszty TYR318. Ten model został wykorzystany do przebadania wpływu mutacji pojedynczych aminokwasów oraz różnych substratów na profil energetyczny.
 
  \begin{subsection}{androst-4-en-3,17-dion}
 
       \begin{tabular}{||c c c c c c||} 
 \hline
 Metoda/Baza & TS1 & I z TS1 & I z TS2 & TS2 & P \\ [0.5ex] 
 \hline\hline
b3lyp/6-31G(d,p) & 12.02 &  1.05 & 1.29 & 12.78 & -3.50 \\
ZPE b3lyp/6-31G(d,p) & -2.70  & -0.66 &  -0.80  & -2.72 & 0.56 \\
PCM(eps = 4) b3lyp/6-31G(d,p) & 15.73  & 8.86 & 7.92 & 20.68 & 3.25 \\
b3lyp/6-311+G(2d,2p) & 12.48  & 1.13  & 1.12 & 13.41 & -4.68 \\
PCM(eps = 4) b3lyp/6-311+G(2d,2p) & 16.34  & 8.72 & 7.61  & 21.04 & 1.79 \\
PCM(eps = 4) b3lyp/6-311+G(2d,2p) & & & & & \\ 
+ ZPE b3lyp/6-31G(d,p)  & 13.65  & 8.10  & 6.82 & 18.31 & 2.35 \\
 \hline
 
\end{tabular}

  
  Pliki z obliczeń:
  \path{/net/archive/groups/plggkatksdh/QM_models/QM_III/AND}
 \end{subsection}
 
 \begin{subsection}{finesteryd}
 
       \begin{tabular}{||c c c c c c||} 
 \hline
 Metoda/Baza & TS1 & I z TS1 & I z TS2 & TS2 & P \\ [0.5ex] 
 \hline\hline
b3lyp/6-31G(d,p) & 18.45 &  4.15 & 3.52 & 11.98 & -6.63 \\
ZPE b3lyp/6-31G(d,p) & -3.08  & -0.48 &  -0.79  & -2.49 & 0.79 \\
PCM(eps = 4) b3lyp/6-31G(d,p) & 22.33  & 13.31 & 13.35 & 21.58 & 1.14 \\
b3lyp/6-311+G(2d,2p) & 17.95  & 4.34  & 3.34 & 11.76 & -9.50 \\
PCM(eps = 4) b3lyp/6-311+G(2d,2p) & 22.14  & 13.55 & 13.11  & 21.20 & -1.85 \\
PCM(eps = 4) b3lyp/6-311+G(2d,2p) & & & & & \\ 
+ ZPE b3lyp/6-31G(d,p)  & 19.05  & 13.07  & 12.32 & 18.71 & -1.06 \\
 \hline
 
\end{tabular}

  
  Pliki z obliczeń:
  \path{/net/archive/groups/plggkatksdh/QM_models/QM_III/finesteryd}
 \end{subsection}
  

 
\end{section}


\begin{section}{Model AcmB QM}

W skład modelu klasterowego AcmB weszły następujące reszty:
\begin{itemize}
 \item Ser49-Gly50-Gly51-Ala52
 \item Tyr115
 \item Tyr118
 \item Thr542-Leu543
 \item Tyr363
 \item Tyr467
 \item Tyr536
\end{itemize}

 \begin{figure}[H]
  \includegraphics[width=0.8\linewidth]{AcmB_I/ts1_tyr.png}
\end{figure}
 
\end{section}

\begin{section}{Przeszukiwanie bazy PDB}
 Przeanalizowano struktury zawarte w Protein Data Bank (PDB) pod kątem występowania podobnych łańcuchów wiązań wodorowych do tych znalezionych w KSTD (a w szczególności zawierających kilka tyrozyn) w pobliżu
 ligandów. Jako ligandy kwalifikowano wszystkie reszty, które nie były aminokwasami, zasadami azotowymi, wodą, prostymi anionami lub jonami metali. Następnie w promieniu 3.5 \angstrom od potencjalnego liganda,
 szukano atomów, które mogły rozpoczynać łańcuch wiązań wodorowych. Jako takie atomy kwalifikowano [nazwa reszty: nazwy atomów]:
 \begin{itemize}
  \item Ser: OG
  \item Cys: SG
  \item Thr: OG1
  \item Asp: OD1, OD2
  \item Lys: NZ
  \item Glu: OE1, OE2
  \item Arg: NH1, NH2
  \item His: ND1, NE2
  \item Tyr: OH
 \end{itemize}

 Następnie w promieniu 3.5 \angstrom wokół każdego z atomów znalezionych w ten sposób szukano kolejnych, które mogą tworzyć z nim wiązanie wodorowe. W ten sposób budowano kolejne sieci wiązań wodorowych.
 Przeanalizowano całą bazę PDB (około 140 tysięcy struktur).
 
 Jeśli chodzi o łańcuchy wiązań wodorowych składających się z samych tyrozyn to znaleziono 1712 sieci typu Tyr-Tyr (1005 plików pdb) oraz 75 sieci typu Tyr-Tyr-Tyr (34 pliki pdb).
 Spośród tych 34 plików pdb zawierających sieć typu Tyr-Tyr-Tyr 12 zawierało również cząsteczke steroidu w sąsiedztwie [plik pdb, reszty tworzące sieć]:
 \begin{itemize}
  \item 2HZQ: A22, A46, A98
  \item 3OWU: A16, A32, A57
  \item 3OWS: A16, A32, A57
  \item 1OH0: A16, A32, A57
  \item 1E3V: A16, A32, A57
  \item 1W6Y: A16, A32, A57
  \item 1CQS: A16, A32, A57
  \item 3OWY: A16, A32, A57
  \item 3FZW: A16, A32, A57
  \item 5KP4: B16, B32, B57
  \item 1OGX: A16, A32, A57
  \item 1E3R: A16, A32, A57
  \item 4C3Y: A119, A318, A487
 \end{itemize}

 Przykład znalezionej struktury z pliku 3OWU:
 
  \begin{figure}[H]
  \includegraphics[width=0.8\linewidth]{proton_relay/3owu.png}
\end{figure}

Enzym ze struktury 3OWU również katalizuje reakcje odworodnienia i podobnie jak w KSTD produkt przejściowy jest stabilizowany przez tautomeryzacje keto enolową i wiązanie wodorowe z jedną z tyrozyn:

 \begin{figure}[H]
  \includegraphics[width=0.8\linewidth]{proton_relay/3owuMechanism.jpg}
\end{figure}


Oczywiście sieci wiązań wodorowych zawierających tyrozyny obok innych reszt jest znacznie więcej. Przykładowo wśród sieci zawierających maksymalnie 4 atomy (186024 sieci, 56098 plików pdb) istnieje 
4125 sieci (2367 plików PDB) zawierających dwie reszty tyrozyny oraz 84 sieci (40 plików pdb) zawierających 3 reszty tyrozyny. 
 
 Oczywiście zastosowana metoda jest dość prymitywna i czasami prowadzi do ciekawych znalezisk. Największa sieć wiązań wodorowych składała się z aż 42 atomów (pdb: 3QPE), sieć widoczna poniżej:
 
   \begin{figure}[H]
  \includegraphics[width=0.8\linewidth]{proton_relay/3qpe.png}
\end{figure}
 
\end{section}

\end{document}
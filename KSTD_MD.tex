\begin{section}{Dynamika molekularna KSTD I}

Cząsteczki występujące w modelu:
\begin{itemize}
 \item białko: 4c3y, pobrane z bazy PDB, łańcuch A
 \item FAD: geometria dostępna ze struktury krystalicznej
 \item AND: androst-4-en-3,17-dion, geometria na podstawie produktu zadokowanego w strukturze krystaliczenej
 \item TYR318: zdeprotonowana tyrozyna, ze względu na brak takiej reszty w oryginalnym polu siłowym amber konieczność parametryzacji
\end{itemize}

Wykorzystane oprogramowanie:
\begin{itemize}
 \item AMBER 16
 \item Gaussian16
 \item RESP
 \item propka
 \item H++
 \item serwer do przygotowywania białka do dynamiki: http://www.playmolecule.org/
 \item moduł pythonowy htmd
\end{itemize}


\begin{subsection}{Przygotowanie obliczeń}

\begin{subsubsection}{Białko}
 pKa aminokwasów zostało oszacowane za pomocą programów: propka3.1 oraz H++. Struktura wykorzystana w obliczeniach zawierała ligandy (FAD oraz AND). 
 
 Poniżej zestawienie wyników (posortowane rosnąco względem różnicy pKa):
 
%   \begin{tabular}{|| c c c c||} 
\begin{longtable}{| p{.25\textwidth} | p{.25\textwidth} | p{.25\textwidth} | p{.25\textwidth} |} 
 \hline
 Reszta & propka & H++ & różnica  \\ [0.5ex] 
 \hline\hline
GLU-87 & 4.22 & 4.170 & 0.04\\
GLU-375 & 2.57 & 2.631 & 0.06\\
ASP-83 & 3.87 & 3.942 & 0.07\\
GLU-215 & 4.65 & 4.727 & 0.07\\
GLU-71 & 3.27 & 3.178 & 0.09\\
GLU-201 & 4.67 & 4.575 & 0.09\\
ASP-138 & 3.43 & 3.319 & 0.11\\
ASP-305 & 3.36 & 3.248 & 0.11\\
GLU-393 & 4.68 & 4.562 & 0.11\\
ASP-330 & 4.01 & 4.131 & 0.12\\
GLU-123 & 4.68 & 4.546 & 0.13\\
GLU-240 & 4.59 & 4.454 & 0.13\\
GLU-7 & 4.04 & 4.182 & 0.14\\
GLU-405 & 4.6 & 4.454 & 0.14\\
ASP-397 & 3.95 & 3.802 & 0.14\\
ASP-470 & 3.04 & 2.862 & 0.17\\
GLU-364 & 4.38 & 4.193 & 0.18\\
ASP-202 & 3.91 & 3.704 & 0.20\\
ASP-372 & 3.93 & 3.702 & 0.22\\
ASP-161 & 2.77 & 2.535 & 0.23\\
GLU-433 & 4.81 & 4.561 & 0.24\\
ASP-144 & 3.25 & 3.510 & 0.25\\
ASP-289 & 2.81 & 2.542 & 0.26\\
GLU-110 & 4.75 & 5.039 & 0.28\\
GLU-62 & 4.73 & 4.425 & 0.30\\
ASP-141 & 4.34 & 4.666 & 0.32\\
HIS-162 & 5.52 & 5.894 & 0.37\\
LYS-510 & 10.63 & 10.246 & 0.38\\
LYS-361 & 10.72 & 11.133 & 0.41\\
GLU-406 & 4.02 & 3.599 & 0.42\\
LYS-186 & 10.38 & 10.803 & 0.42\\
GLU-345 & 4.05 & 3.613 & 0.43\\
GLU-192 & 4.87 & 4.405 & 0.46\\
ASP-386 & 3.52 & 3.992 & 0.47\\
LYS-400 & 10.55 & 11.122 & 0.57\\
ASP-154 & 4.31 & 3.723 & 0.58\\
GLU-411 & 4.8 & 5.387 & 0.58\\
ASP-277 & 3.62 & 3.029 & 0.59\\
GLU-94 & 4.13 & 3.500 & 0.62\\
ASP-465 & 2.19 & 2.823 & 0.63\\
GLU-376 & 4.79 & 4.155 & 0.63\\
ASP-9 & 2.89 & 2.162 & 0.72\\
TYR-502 & 10.93 & 11.684 & 0.75\\
LYS-148 & 10.64 & 11.396 & 0.75\\
GLU-237 & 4.28 & 5.044 & 0.76\\
GLU-188 & 4.05 & 3.245 & 0.80\\
GLU-209 & 4.79 & 3.967 & 0.82\\
ASP-127 & 4.0 & 3.142 & 0.85\\
LYS-121 & 10.57 & 11.446 & 0.87\\
ASP-136 & 3.85 & 2.971 & 0.87\\
ASP-326 & 4.94 & 4.012 & 0.92\\
GLU-309 & 5.2 & 6.151 & 0.95\\
LYS-380 & 10.45 & 11.412 & 0.96\\
ASP-329 & 3.39 & 2.427 & 0.96\\
GLU-112 & 4.21 & 3.234 & 0.97\\
GLU-211 & 4.2 & 5.185 & 0.98\\
ASP-40 & 3.81 & 2.807 & 1.00\\
LYS-247 & 10.14 & 11.241 & 1.10\\
ASP-507 & 3.93 & 2.678 & 1.25\\
ASP-261 & 3.07 & 1.728 & 1.34\\
GLU-286 & 4.64 & 3.127 & 1.51\\
HIS-408 & 6.84 & 8.407 & 1.56\\
GLU-104 & 2.84 & 1.269 & 1.57\\
ASP-90 & 2.39 & 0.691 & 1.69\\
GLU-85 & 4.06 & 2.323 & 1.73\\
ASP-415 & 5.68 & 3.872 & 1.80\\
HIS-362 & 4.3 & 6.239 & 1.93\\
HIS-328 & 6.97 & 9.197 & 2.22\\
ASP-68 & 2.55 & 0.301 & 2.24\\
ASP-342 & 3.11 & 0.837 & 2.27\\
ASP-456 & 3.64 & 0.561 & 3.07\\
GLU-152 & 4.99 & 1.029 & 3.96\\
ARG-344 & 12.47 & $>$12.000 & -\\
ARG-299 & 13.5 & $>$12.000 & -\\
TYR-414 & 14.02 & $>$12.000 & -\\
TYR-24 & 10.45 & $>$12.000 & -\\
TYR-473 & 13.47 & $>$12.000 & -\\
ASP-412 & 2.87 & $<$0.000 & -\\
ASP-446 & 4.11 & $<$0.000 & -\\
ARG-239 & 12.77 & $>$12.000 & -\\
ARG-150 & 13.86 & $>$12.000 & -\\
TYR-318 & 22.84 & $>$12.000 & -\\
ARG-176 & 11.28 & $>$12.000 & -\\
ARG-323 & 13.07 & $>$12.000 & -\\
CYS-282 & 13.47 & $>$12.000 & -\\
ARG-310 & 15.29 & $>$12.000 & -\\
ASP-118 & 5.33 & $<$0.000 & -\\
CYS-8 & 12.04 & $>$12.000 & -\\
GLU-233 & 6.22 & $<$0.000 & -\\
ARG-63 & 13.98 & $>$12.000 & -\\
TYR-119 & 17.57 & $>$12.000 & -\\
TYR-438 & 10.14 & $>$12.000 & -\\
ASP-404 & 1.74 & $<$0.000 & -\\
ARG-41 & 12.89 & $>$12.000 & -\\
ARG-114 & 13.08 & $>$12.000 & -\\
ARG-389 & 14.35 & $>$12.000 & -\\
TYR-120 & 12.7 & $>$12.000 & -\\
TYR-48 & 13.05 & $>$12.000 & -\\
ARG-223 & 12.6 & $>$12.000 & -\\
ARG-463 & 12.33 & $>$12.000 & -\\
ARG-204 & 13.05 & $>$12.000 & -\\
CYS-353 & 11.83 & $>$12.000 & -\\
TYR-487 & 14.21 & $>$12.000 & -\\
ARG-125 & 13.69 & $>$12.000 & -\\
GLU-37 & 3.15 & $<$0.000 & -\\
TYR-311 & 12.66 & $>$12.000 & -\\
ARG-218 & 12.34 & $>$12.000 & -\\
ARG-78 & 13.15 & $>$12.000 & -\\
ARG-74 & 15.6 & $>$12.000 & -\\
ARG-171 & 10.56 & $>$12.000 & -\\
ARG-409 & 11.51 & $>$12.000 & -\\
LYS-38 & 10.13 & $>$12.000 & -\\
LYS-220 & 10.54 & $>$12.000 & -\\
ASP-156 & 3.57 & $<$0.000 & -\\
LYS-450 & 6.49 & $>$12.000 & -\\
ARG-88 & 11.39 & $>$12.000 & -\\
ARG-503 & 11.77 & $>$12.000 & -\\
ARG-441 & 12.83 & $>$12.000 & -\\
TYR-76 & 13.96 & $>$12.000 & -\\
GLU-314 & 3.46 & $<$0.000 & -\\
CYS-419 & 10.57 & $>$12.000 & -\\
ARG-157 & 10.86 & $>$12.000 & -\\
LYS-394 & 11.38 & $>$12.000 & -\\
ARG-460 & 13.64 & $>$12.000 & -\\
ASP-319 & 4.46 & $<$0.000 & -\\
ARG-130 & 14.06 & $>$12.000 & -\\
ARG-190 & 12.74 & $>$12.000 & -\\
ARG-485 & 12.34 & $>$12.000 & -\\
TYR-92 & 14.51 & $>$12.000 & -\\
 \hline
 \end{longtable}
% \end{tabular}
 
 Należy zauważyć, że w przypadku TYR318 propka sugeruje pKa na poziomie 22.84 -- jest to sprzeczne z postulowanym mechanizme, ale należy pamiętać, że propka 
 uwzględnia ligandy w obliczeniach w sposób bardzo przybliżony. Przewidywane pI przez program propKa (4.40) jest w bardzo dobrej zgodności z eksperymentalnym (4.70).
 
 Aby ostatecznie przygotować cząsteczke białka do symulacji MD posłużono się narzędziem: PlayMolecule ProteinPrepare (publikacja: "PlayMolecule ProteinPrepare:
 A Web Application for Protein Preparation for Molecular Dynamics Simulations." Martinez-Rosell, Giorgino, De Fabritiis ).
 Narzędzie to wykorzystuje propke oraz PDB2PQR (optymalizacja wiązań wodorowych w białku) jako programy pomocnicze. Korzystając z tego i wyników uzyskanych niezależnie z propki i H++
 ustalono stany protonacyjne aminokwasów w cząsteczce.
 
 Istotne pliki:
 
\begin{itemize}
 \item \path{/home/glanowski/docs/KSTD_MD_I/proteinPrepare/monomer_substrate_final.pdb} -- struktura łańcucha A 4c3y
 \item \path{/home/glanowski/docs/KSTD_MD_I/proteinPrepare/monomer_substrate_final.pka} -- wyniki z propKi
 \item \path{/home/glanowski/docs/KSTD_MD_I/proteinPrepare/h++.log} -- wyniki z H++
 \item \path{/home/glanowski/docs/KSTD_MD_I/proteinPrepare/propKavsHpp.py} -- skrypt do porównania wyników
\end{itemize}

\end{subsubsection}

\begin{subsubsection}{FAD}

Ładunki dla cząsteczki FAD zostąły pobrane z bazy R.E.DD.B. (https://upjv.q4md-forcefieldtools.org/REDDB/projects/F-91/, kopia projektu: \path{/home/glanowski/docs/KSTD_MD_I/FAD/F-91} ).
Szczegóły obliczeniowe wg autorów:

\begin{verbatim}
Geometry optimization

Program 1 GAUSSIAN 2003
Theory level 1 HF
More information 1 Tight
Basis set 1 6-31G*

Molecular electrostatic potential computation

Program 2 GAUSSIAN 2003
Theory level 2 DFT B3LYP
More information 2 IOp(6/33=2) SCRF(IEFPCM,Solvent=Ether) NoSymm
Basis set 2 cc-pVTZ

Algorithm CONNOLLY SURFACE

Information about the charge fit

Program RESP
Number of stage(s) 2
\end{verbatim}

Po pierwszym uruchomieniu obliczeń wprowadzono jedną modyfikacje w pliku frcmod: zwiększono wartość improper:\\
\begin{verbatim}
 h5-nb-ca-nb        10.5          180.0         2.0
\end{verbatim}
aby zapobiec wygięciu pierścieni w cząsteczce FAD.

Ładunki cząstkowe atomów w cząsteczce FAD:
 \begin{figure}[H]
  \includegraphics[width=0.8\linewidth]{MD_KSTD_I/FAD/FAD_charges1.png}
\end{figure}

 \begin{figure}[H]
  \includegraphics[width=0.8\linewidth]{MD_KSTD_I/FAD/FAD_charges2.png}
\end{figure}

 \begin{figure}[H]
  \includegraphics[width=0.8\linewidth]{MD_KSTD_I/FAD/FAD_charges3.png}
\end{figure}

 \begin{figure}[H]
  \includegraphics[width=0.8\linewidth]{MD_KSTD_I/FAD/FAD_charges4.png}
\end{figure}

 \begin{figure}[H]
  \includegraphics[width=0.8\linewidth]{MD_KSTD_I/FAD/FAD_charges5.png}
\end{figure}
 
\end{subsubsection}

\begin{subsubsection}{AND}
Cząsteczka androst-4-en-3,17-dionu została zoptymalizowana w próżni (gaussian09 b3lyp/6-31g(d)). Dla zoptymalizowanej struktury
przeprowadzono obliczenia ESP. Następnie za pomocą programu antechamber(RESP) przekonwertowano uzyskany plik gesp na plik mol2, zwierający ładunki atomów.
Utworzono również pliki prepi oraz frcmod.
Pliki z obliczeń:
\path{/home/glanowski/docs/KSTD_MD_I/AND}

 
\end{subsubsection}

\begin{subsubsection}{Zdeprotonowana tyrozyna}

Aby obliczyć ładunki zdeprotonowanej tyrozyny, zoptymalizowano geometrie (gaussian09 b3lyp/6-31g(d)) następującej cząsteczki:

\begin{figure}[H]
  \includegraphics[width=0.5\linewidth]{MD_KSTD_I/TYM/tym_capped.png}
\end{figure}

Następnie wykorzystano program RESP z opcją wymuszenia ładunków na N i C końcu peptydu (w ten sposób sumaryczny ładunek reszty będzie równy dokładnie -1):

\begin{figure}[H]
  \includegraphics[width=0.5\linewidth]{MD_KSTD_I/TYM/ace_with_charges.jpg}
\end{figure}

\begin{figure}[H]
  \includegraphics[width=0.5\linewidth]{MD_KSTD_I/TYM/nme_with_charges.jpg}
\end{figure}

Ładunki cząstkowe na reszcie zdeprotonowanej tyrozyny:
\begin{figure}[H]
  \includegraphics[width=0.8\linewidth]{MD_KSTD_I/TYM/charges.png}
\end{figure}

Warto zauważyć, że ładunki na równocennych atomach są identyczne -- czyli tak jak powinno być.

Źródło: \path{http://ambermd.org/tutorials/advanced/tutorial1/section1.htm}
Pliki z obliczeń:
\path{/home/glanowski/docs/KSTD_MD_I/TYM}

\end{subsubsection}

 
\end{subsection}

\begin{subsection}{Obliczenia}
Cały model KSTD został najpierw zminimalizowany w 5 krokach. W pierwszych 4 krokach na atomy białka i ligandów został nałożony dodatkowy potencjał mający ograniczyć ich ruch, 
tak aby najpierw zoptymalizować wygenerowane cząsteczki wody. Stałe siłowe wynosiły kolejno: 500, 250, 100 i 10 $\frac{kcal}{mol \cdot \angstrom^2 }$.

Następnie model został ogrzany do 303 K w czasie 100 ps, w warunkach stałej objętości. Na białko i ligandy nałożony został nieznaczny dodatkowy potencjał
1 $\frac{kcal}{mol \cdot \angstrom^2 }$ aby zabezpieczyć ich strukturę. W kolejnym kroku przeprowadzono 100 ps dynamiki pod stałym ciśnieniem z identycznym dodatkowym potencjałem, w 
temperaturze 303 K. W ten sposób przygotowano model do właściwej symulacji dynamiki molekularnej (stałe ciśnienie, 303 K).

Ostatecznie przeprowadzono 132 ns symulacji, jednakże już po upływie 45 ns wybrano jej fragment do dalszej analizy. 20 nanosekund (oznaczone zieloną ramką) poddano analizie skupień.
Zastosowano metodę k-średnich. Do inicjalizacji analizy skupień wybrano w sposób losowy co dziesiątą klatke. 
Liczba klastrów wynosiła 5, ostatecznie wybrano strukture leżącą najbliżej centroidu najczęśćiej występującego skupiska.



\begin{figure}[H]
  \includegraphics[width=0.5\linewidth]{MD_KSTD_I/rmsdGutSel.png}
\end{figure}

\end{subsection}

\end{section}
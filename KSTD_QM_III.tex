\begin{section}{Model QM III}
 Najistotniejszą zmianą w stosunku do poprzedniego modelu klasterowego KSTD było uwzględnienie fragmentu (wiązanie peptydowe) LEU494, który mógłby stabilizować cząsteczke FADu (zwłaszcza po przyłączeniu jonu
 $H^{-}$). Wzięto pod uwage również nieco więcej atomów z reszty TYR318. Ten model został wykorzystany do przebadania wpływu mutacji pojedynczych aminokwasów oraz różnych substratów na profil energetyczny.
 
  \begin{subsection}{androst-4-en-3,17-dion}
 
       \begin{tabular}{||c c c c c c||} 
 \hline
 Metoda/Baza & TS1 & I z TS1 & I z TS2 & TS2 & P \\ [0.5ex] 
 \hline\hline
b3lyp/6-31G(d,p) & 12.02 &  1.05 & 1.29 & 12.78 & -3.50 \\
ZPE b3lyp/6-31G(d,p) & -2.70  & -0.66 &  -0.80  & -2.72 & 0.56 \\
PCM(eps = 4) b3lyp/6-31G(d,p) & 15.73  & 8.86 & 7.92 & 20.68 & 3.25 \\
b3lyp/6-311+G(2d,2p) & 12.48  & 1.13  & 1.12 & 13.41 & -4.68 \\
PCM(eps = 4) b3lyp/6-311+G(2d,2p) & 16.34  & 8.72 & 7.61  & 21.04 & 1.79 \\
PCM(eps = 4) b3lyp/6-311+G(2d,2p) & & & & & \\ 
+ ZPE b3lyp/6-31G(d,p)  & 13.65  & 8.10  & 6.82 & 18.31 & 2.35 \\
 \hline
 
\end{tabular}

  
  Pliki z obliczeń:
  \path{/net/archive/groups/plggkatksdh/QM_models/QM_III/AND}
 \end{subsection}
 
 \begin{subsection}{finesteryd}
 
       \begin{tabular}{||c c c c c c||} 
 \hline
 Metoda/Baza & TS1 & I z TS1 & I z TS2 & TS2 & P \\ [0.5ex] 
 \hline\hline
b3lyp/6-31G(d,p) & 18.45 &  4.15 & 3.52 & 11.98 & -6.63 \\
ZPE b3lyp/6-31G(d,p) & -3.08  & -0.48 &  -0.79  & -2.49 & 0.79 \\
PCM(eps = 4) b3lyp/6-31G(d,p) & 22.33  & 13.31 & 13.35 & 21.58 & 1.14 \\
b3lyp/6-311+G(2d,2p) & 17.95  & 4.34  & 3.34 & 11.76 & -9.50 \\
PCM(eps = 4) b3lyp/6-311+G(2d,2p) & 22.14  & 13.55 & 13.11  & 21.20 & -1.85 \\
PCM(eps = 4) b3lyp/6-311+G(2d,2p) & & & & & \\ 
+ ZPE b3lyp/6-31G(d,p)  & 19.05  & 13.07  & 12.32 & 18.71 & -1.06 \\
 \hline
 
\end{tabular}

  
  Pliki z obliczeń:
  \path{/net/archive/groups/plggkatksdh/QM_models/QM_III/finesteryd}
 \end{subsection}
  

 
\end{section}